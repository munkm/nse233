\documentclass[11pt, a4paper]{article}
\usepackage[inner=1in,outer=1in,top=1in,bottom=1in]{geometry}
\pagestyle{empty}
\usepackage{placeins}
\usepackage{graphicx}
\usepackage{fancyhdr, lastpage, bbding, pmboxdraw}
\usepackage{amsmath, amssymb}
\usepackage[usenames,dvipsnames]{color}
\definecolor{darkblue}{rgb}{0,0,.6}
\definecolor{darkred}{rgb}{.7,0,0}
\definecolor{darkgreen}{rgb}{0,.6,0}
\definecolor{red}{rgb}{.98,0,0}
\usepackage[colorlinks,pdfusetitle,urlcolor=darkblue,citecolor=darkblue,linkcolor=darkred,bookmarksnumbered,plainpages=false]{hyperref}
%\renewcommand{\thefootnote}{\fnsymbol{footnote}}

\pagestyle{fancyplain}
\fancyhf{}
\lhead{ \fancyplain{}{\CourseTitle} }
%\chead{ \fancyplain{}{} }
\rhead{ \fancyplain{}{\CourseSemester \CourseYear} }
%\rfoot{\fancyplain{}{page \thepage\ of \pageref{LastPage}}}
\fancyfoot[RO, LE] {page \thepage\ of \pageref{LastPage} }
\thispagestyle{plain}
\usepackage{tabularx}


%%%%%%%%%%%%%%%%%%%%%%%%%%%%%%%%%%%%
\usepackage{xspace}

\newcommand{\CourseNumber}{NSE 233}
\newcommand{\CourseTitle}{Mathematical Methods for Nuclear Science and Engineering\xspace}%
\newcommand{\CourseInstructor}{Dr. Madicken Munk\xspace}%
\newcommand{\CourseSemester}{Spring\xspace}%
\newcommand{\CourseYear}{2025\xspace}%
\newcommand{\CourseDays}{M\xspace}%
\newcommand{\CourseStart}{11:00\xspace}%
\newcommand{\CourseEnd}{11:50\xspace}%
\newcommand{\CourseInstructorEmail}{madicken.munk@oregonstate.edu}
\newcommand{\CourseRoom}{122\xspace}%
\newcommand{\CourseBuilding}{Crop Science Building\xspace}%
\newcommand{\CourseZoom}{}%
\newcommand{\CourseUniversity}{Oregon State University\xspace}%
\newcommand{\TeachingAssistant}{Eric Hudec\xspace}%
\newcommand{\TAOfficeHourDays}{Date Unknown \xspace}%
\newcommand{\TAOfficeHourStart}{TBA\xspace}%
\newcommand{\TAOfficeHourEnd}{TBA\xspace}%
\newcommand{\TAOfficeHourPlace}{205 Merryfield Hall\xspace}
\newcommand{\MunkOfficeHourDays}{Mondays\xspace}%
\newcommand{\MunkOfficeHourStart}{9:00 a.m.\xspace}%
\newcommand{\MunkOfficeHourEnd}{9:50 a.m.\xspace}%
\newcommand{\MunkOfficeHourPlace}{OSU Radiation Center Library\xspace}
%\newcommand{\Course<++>}{<++>}
%\newcommand{\Course<++>}{<++>}
%%%%%%%%%%%%%%%%%%%%%%%%%%%%%%%%%%%%
\title{\CourseNumber: \CourseTitle\\}
\author{\CourseUniversity}
\date{\CourseSemester \CourseYear}
\begin{document}
\maketitle
%\setlength{\unitlength}{1in}
\renewcommand{\arraystretch}{1.5}
\begin{center}
\begin{table}[h]
\begin{tabularx}{\textwidth}{rXrX}
\hline
\textbf{Instructor:} & \CourseInstructor & \textbf{Email:} & \href{mailto:\CourseInstructorEmail}{\CourseInstructorEmail} \\
\textbf{Time:} & \CourseDays \CourseStart -- \CourseEnd & \textbf{Place:} & \CourseRoom \CourseBuilding \\
\textbf{TA:} & \TeachingAssistant & & \\
\hline
\end{tabularx}
\end{table}
\end{center}

\paragraph{Course Pages:}
\begin{enumerate}
        \item \url{https://canvas.oregonstate.edu/courses/2002879}
        \item \url{https://github.com/munkm/nse233}
\end{enumerate}

\paragraph{TA Office Hours:} \TeachingAssistant is the TA for the course and will hold
office hours in \TAOfficeHourPlace.
They will hold office hours \TAOfficeHourDays from
\TAOfficeHourStart to \TAOfficeHourEnd.
Grade disputes will not be addressed in TA office hours.

\paragraph{Office Hours:} Dr. Munk  will hold office hours 
% on
% \MunkOfficeHourDays from \MunkOfficeHourStart to \MunkOfficeHourEnd in
% \MunkOfficeHourPlace. Supplemental office hours are 
by appointment only
and should be requested with at least 24 hours notice.
Before making an appointment, please try the following options:
\begin{itemize}
\item If your colleagues might be helpful, please post your questions in the
        canvas discussion forum provided for this purpose.
\item If the TAs might be helpful, please attend their office hours.
\item Email Dr. Munk. If possible, please phrase your question such that it
        can be answered `Yes' or `No'. This helps ensure that I can reply quickly 
        and help you faster. 
\end{itemize}

If none of the above are successful or appropriate, you may email me with a
selection of times of your availability and we can find a time that is mutually
agreeable.

\paragraph{Main References:}
The required text for this course is \cite{mcclarren_computational_2017}. This textbook is available for free by download through the OSU Library. It is also in the modules section of the course for you to download. 

For a supplementary textbook for the python basics of this course, \cite{scopatz_effective_2015} is available online and in the library as an ebook.


\bibliographystyle{unsrt}
\renewcommand{\refname}{\normalfont\selectfont\normalsize}\vspace{-1cm}
\bibliography{nse233}

\paragraph{Course Description:}
Development and application of analytical and numerical methods with 
applications to problems in the field of Nuclear Science and Engineering. 
Major topics will include solution of ODEs and systems of ODEs, root finding techniques and numerical integration and differentiation. Major applications will
include solution of the Batemann Equations and solution of the diffusion equation. 

\paragraph{Course Learning Objectives:}

This course will introduce students to numerical and analytical solution
techniques for problems found in nuclear science and engienering. Students
will demonstrate the ability to: 

\begin{enumerate}
\item Solve ODEs using analytical techniques
\item Solve ODEs using numerical techniques
\item Solve systems of ODEs using numerical techniques
\item Implement numerical differentiation and integration algorithms
\item Implement numerical root finding techniques
\item Implement basic Monte Carlo based algorithms. 
\item Understand the accuracy of various numerical techniques. 
\end{enumerate}

\paragraph{Course Topics:}
\noindent Topics that will be covered in this course include:
\begin{enumerate}
\item Review of Python basics,
\item Interpolation methods, polynomial curve fit, spline curve fit,
\item Use of the integrating factor, solution of homogenous and inhomogenous ODEs, 
\item Solution of systems of ODEs,
\item Linear algebra,
\item Root finding techniques,
\item Numerical integration,
\item Numerical differentiation, and
\item Introduction to Monte Carlo techniques. 
\end{enumerate}

\paragraph{Prerequisites:}
\begin{itemize}
\item MTH 256 or 256H with a C or better. 
\end{itemize}

\paragraph{Grading Policy:} Grades will be assigned as a weighted sum of the following work:

\begin{table}[h]
\begin{tabularx}{\textwidth}{Xr}
\textbf{Work} & \textbf{Weight}\\
\hline
\textbf{Homework} & (70\%) \\
\textbf{Weekly Quizzes} & (10\%) \\
\textbf{Final Project} & (20\%) \\
\hline
\textbf{Total} & (100\%) \\
\end{tabularx}
\end{table}

\paragraph{Important Dates:}
\begin{center} \begin{minipage}{3.8in}
\begin{flushleft}
Final Project Deadline      \dotfill Sunday, June 08, 11:59pm  \\
\end{flushleft}
\end{minipage}
\end{center}

\paragraph{Recommended competency:}
\begin{itemize}
\item Computer programming experience. 
\end{itemize}

\paragraph{Class Policies:}

\begin{itemize}
\item[] \textbf{Integrity:} This is an institution of higher
learning. You are expected to uphold the principles of academic 
integrity at our institution and to be honest and ethical with your academic work. 
Academic dishonesty such as plagiarism, cheating, assisting, tampering, and falsification will not be 
tolerated. 
For reference, please note the \href{https://studentlife.oregonstate.edu/studentconduct/academic-misconduct-students}{Student Guide for Academic Misconduct} 
on OSU's Dean of Students page. 

\item[] \textbf{Attendance:} Regular attendance and full participation in class sessions is expected. 

\item[] \textbf{Electronics:} Active participation is essential and expected. 
  You are encouraged to follow along with course jupyter notebook content with 
  your personal laptops during class. Use of cell phones and computers for 
  engagement with material not relevant to the class is not permitted. 

\item[] \textbf{Collaboration:} Collaboratively reviewing course materials and
  working fellow students can be enriching.  This is
  recommended.  However, unless otherwise instructed, homework assignments, quizzes, and the final project are
  to be completed independently and 
  the result of one's own independent work. Dr. Munk recommends working through
  the homework problems independently and then checking work with peers. Explaining your
  process is a good exercise to retain course material.

\item[] \textbf{Late Work:} Late work will not be accepted. However, we will drop your lowest scoring homework and quiz from your grade. 
This shall accomodate unforseen circumstances that may contribute to a missed quiz or homework assignment.

\item[] \textbf{Make-up Work:} There will be no negotiation about late work
        except in the case of absence documented by an 
\href{https://studentlife.oregonstate.edu/emergency-notifications}{emergency notification} from the Dean of Students.
        Dean of Students. Please note that such a letter is appropriate for many
        types of conflicts. 

\item[] \textbf{Grade Disputes:} It is important that you understand and agree
        with the grade you receive on your reports. If you would like
        to dispute your score, you must send an explanation by email to Dr.
        Munk within one week of recieving the grade.
        \textbf{Do not expect us to regrade anything while in conversation with
        you} as that would not be fair to the other students in the class, whose
        homeworks were graded without them present.  If you request a regrade,
        be aware that the it is possible that your score will go down.
        Regrade requests should be based on an error on our part (e.g., adding
        up the points incorrectly) or what you suspect is a misunderstanding of
        your work (e.g., arriving at the correct answer using an unexpected
        technique). Regrade requests that argue with the rubric (e.g., ``this is
        wrong, but you took too many points off'') will be returned without
        consideration.
        \textbf{Your work should stand alone.} If an assignment is disorganized or
        ambiguous, and requires an extensive explanation to the TA, you
        will likely still lose points. 
        Your writing and solutions should be your own and not obtained from another
        source or generative tool. 
        The assignments not only evaluate your
        understanding of the material - they also evaluate your ability to
        communicate that understanding clearly.
\end{itemize}

\paragraph{University-Wide Course Statements:}

\begin{itemize}
\item[] \textbf{Academic Calendar:} All students are subject to the 
registration and refund deadlines as stated in the 
\href{https://registrar.oregonstate.edu/osu-academic-calendar}{Academic Calendar}

\item[] \textbf{Statement Regarding Students with Disabilities:} 
Accommodations for students with disabilities are determined and approved 
by Disability Access Services (DAS). If you, as a student, believe you are 
eligible for accommodations but have not obtained approval please contact 
DAS immediately at 541-737-4098 or at \url{http://ds.oregonstate.edu/}. 
DAS notifies students and faculty members of approved academic accommodations 
and coordinates implementation of those accommodations. While not required, 
students and faculty members are encouraged to discuss details of the 
implementation of individual accommodations.

\item[] \textbf{Student Conduct Expectations Link:} 
\url{https://beav.es/codeofconduct}

\item[] \textbf{Student Bill of Rights:} 
OSU has 
\href{https://asosu.oregonstate.edu/advocacy/rights}{twelve established student rights}. 
They include due process in all 
university disciplinary processes, an equal opportunity to learn, and grading 
in accordance with the course syllabus. 

\item[] \textbf{Reach out for Success:} 
University students encounter setbacks from time to time. If you encounter
 difficulties and need assistance, it’s important to reach out. Consider 
discussing the situation with an instructor or academic advisor. Learn about 
resources that assist with wellness and academic success at
\url{https://oregonstate.edu/ReachOut}. 
If you are in immediate crisis, please contact the Crisis Text Line by texting 
OREGON to 741-741 or call the National Suicide Prevention Lifeline at 
1-800-273-TALK (8255). 
\end{itemize}


\paragraph{Accessibility:} I hope that this course will be inclusive and
accommodating for all learners. If you have worked with DAS to obtain 
accomodations, please reach out so we can work together to ensure you 
are appropriately accomodated. 

\paragraph{Safety:}
Emergencies can happen anywhere and at any time, so it’s important that we take
a minute to prepare for a situation in which our safety could depend on our
ability to react quickly. Take a moment to learn the different ways to leave
this building. If there's ever a fire alarm or something like that, you’ll know
how to get out and you'll be able to help others get out. Next, figure out the
best place to go in case of severe weather - we'll need to go to a low-level in
the middle of the building, away from windows. And finally, if there's ever
someone trying to hurt us, our best option is to run out of the building. If we
cannot do that safely, we'll want to hide somewhere we can't be seen, and we'll
have to lock or barricade the door if possible and be as quiet as we can. 
If we can't run or hide, we'll fight back with whatever we can
get our hands on. 
Remember you can sign up for
emergency text messages at \url{https://emergency.oregonstate.edu/emergency-management/osu-alerts}. This
\href{https://emergency.oregonstate.edu/emergency-preparedness/emergency-procedures/run-hide-fight-info}{Run Hide Fight Info page}
discusses the OSU's Run-Hide-Fight strategy.


\paragraph{Other Resources:}
University students typically experience a wide range of stressors during their
time on campus. Accordingly, campus resources exist to help students manage
stress levels, mental health, physical health, and emergencies while navigating
this environment. I hope you will take advantage of these campus resources.

\begin{itemize}
\item \href{https://recsports.oregonstate.edu/}{Campus Recreational Sports}
\item \href{https://counseling.oregonstate.edu/}{Counselling and Psychological Services}
\item \href{https://studenthealth.oregonstate.edu/}{Student Health Services}
\item \href{https://studentlife.oregonstate.edu/bnc}{The Basic Needs Center}
\end{itemize}

\pagebreak
\FloatBarrier
\renewcommand{\arraystretch}{1}
\begin{table}[h]
\begin{center}
\begin{tabular}{lllcccccc}
\multicolumn{8}{c}{\textbf{Course Schedule:}\textit{ Note that this schedule is
subject to change.}}\\
&&&&&&&&\\
\textbf{Date} & \textbf{Week} & \textbf{Day} & \textbf{Unit} & \textbf{Quiz} & \textbf{Quiz}& \textbf{Homework} & \textbf{Homework}\\
              &  &  &  & \textbf{Given}  & \textbf{Due\footnotemark[1]}  & \textbf{Given} & \textbf{Due\footnotemark[2]}\\ \hline
\hline
3/31 & 1 & M & Intro/Syllabus                     &   &   &   &  \\
4/2  & 1 & W & Python Review                      & 1 &   & 1 &  \\
4/4  & 1 & F & --- No Class ---                   &   &   &   &  \\
4/7  & 2 & M & Data Structures and Precision      &   & 1 &   &  \\
4/9  & 2 & W & Functions                          &   &   & 2 & 1 \\
4/11 & 2 & F & Recursion and Modules              & 2 &   &   &  \\
4/14 & 3 & M & Numpy and Array Operations         &   & 2 &   &  \\
4/16 & 3 & W & Matplotlib and Plotting            &   &   &   &  \\
4/18 & 3 & F & Dictionaries, Testing              & 3 &   &   &  \\
4/21 & 4 & M & Gaussian Elimination               &   & 3 & 3 & 2\\
4/23 & 4 & W & LU Factorization                   &   &   &   &  \\
4/25 & 4 & F & Iterative Methods                  & 4 &   &   &  \\
4/28 & 5 & M & --- No Class ---                   &   & 4 &   &  \\
4/30 & 5 & W & Interpolation                      &   &   & 4 & 3\\
5/2  & 5 & F & Interpolation                      & 5 &   &   &  \\
5/5  & 6 & M & Curve Fitting                      &   & 5 &   &  \\
5/7  & 6 & W &        Root Finding Methods        &   &   &   &  \\
5/9  & 6 & F & Root Finding Methods               & 6 &   & 5 & 4\\
5/12 & 7 & M & Root Finding Methods               &   & 6 &   &  \\
5/14 & 7 & W & Finite Different Approximations    &   &   &   &  \\
5/16 & 7 & F & Numerical Integration              & 7 &   &   &  \\
5/19 & 8 & M & Initial Value Problems             &   & 7 & 6 & 5\\
5/21 & 8 & W & Systems of ODEs                    &   &   &   &  \\
5/23 & 8 & F & Point Reactor Kinetics             & 8 &   &   &  \\
5/26 & 9 & M & --- Memorial Day ---               &   &   &   &  \\
5/28 & 9 & W & Intro to Monte Carlo Methods       &   & 8 & 7 & 6\\
5/30 & 9 & F & Tracking and Sampling              & 9 &   &   &  \\
6/2  & 10 & M & Variance Reduction                &   & 9 &   &  \\
6/4  & 10 & W & Variance Reduction                &   &   &   &  \\
6/6  & 10 & F & Buffer                            &   &   &   & 7\\
\end{tabular}
\end{center}
\end{table}
\FloatBarrier


\footnotetext[1]{Quizzes will be due at 11am the day the quiz is due.}
\footnotetext[2]{Homework is due at 11:59 PM pacific the the date it is due.}
%%%%%% THE END
\end{document}
